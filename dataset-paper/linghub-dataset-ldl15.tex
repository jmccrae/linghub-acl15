\documentclass[11pt]{article}
\usepackage{acl2015}
\usepackage[utf8]{inputenc}
\usepackage{times}
\usepackage{url}
\usepackage{latexsym}

\usepackage{makecell}
\usepackage{pbox}
\usepackage{color}
\usepackage{graphicx}

\setlength\titlebox{6cm}

% You can expand the titlebox if you need extra space
% to show all the authors. Please do not make the titlebox
% smaller than 5cm (the original size); we will check this
% in the camera-ready version and ask you to change it back.


\title{Linghub: Aggregated Metadata about Language Resources as Linked Data}

\author{John P. McCrae, Philipp Cimiano \\
  CIT-EC, Bielefeld University \\
  Bielefeld, Germany \\
  {\scriptsize\tt\{jmccrae, cimiano\}@cit-ec.uni-bielefeld.de}}
%\\
%  {\bf Victor Rodr\'iguez Doncel, Daniel Vila-Suero}\\
%  {\bf Jorge Gracia} \\
%  Universidad Polit\'ecnica de Madrid \\
%  Madrid, Spain \\
%  {\scriptsize\tt\{vrodriguez, dvila, jgracia\}@fi.upm.es} \\\And
%  Luca Matteis, Roberto Navigli \\
%  University of Rome, La Sapienza \\
%  Rome, Italy \\
%  {\scriptsize\tt \{matteis, navigli\}@di.uniroma1.it} \\
%  {\bf Andrejs Abele, Gabriela Vulcu}\\
%  {\bf Paul Buitelaar} \\
%  Insight Centre, National University of Ireland\\
%  Galway, Ireland \\
%  {\scriptsize\tt \{andrejs.abele, gabriela.vulcu,}\\
%  {\scriptsize\tt paul.buitelaar\}@insight-centre.org} \\}
 
\date{}

\begin{document}
\maketitle
\begin{abstract}
    Abstract.
\end{abstract}

\section{Introduction}

Language resources are essential for nearly all tasks in natural language
processing (NLP) and in particular
for the adaptation of resources and methods to new domains and languages. In
order to use language resources for new purposes they must first be discovered
and this can only be done if there is a comprehensive list of all resources that
may be available. To this there have been a number of projects that have
attempted to collect such a catalogue using various methods and with differing
degrees of data quality. We present a new portal, Linghub, that aims to integrate all
these data from different sources by means of linked data and thus to create a
portal, whereby all information about language resources can be included and
queried using a common methodology. As such, this resource will enable wider
discovery of language resources for researchers in NLP, computational
linguistics and linguistics.

Currently, the approaches to metadata collection can be split into two broad
classes: firstly, \emph{curatorial} resources, which are those for which collections of
language resources are maintained by one or more institute. Such resources have
an advantage in that such metadata is normally of very high quality, however the
resulting data often fails to cover the whole spectrum of data available.
Examples of this include the META-SHARE~\cite{federmann2012meta} project and the
CLARIN project's Virtual Language Observatory~\cite[VLO]{van2012semantic}. On
the other hand, \emph{collaborative} approaches rely on data publishers
self-reporting data about their own language resources. This can be advantageous
as it allows reporting by researchers not directly collected to existing
infrastructure projects, however the resulting data is often of lower quality as
the systems may use free-text input or tagging input rather than controlled
vocabularies, as they are easier for non-expert users to understand.

Given the nature of this difference we wish to make data available from multiple
sources in a homogeneous manner and to this end we adopted a model based on the
DCAT data model~\cite{maali2014data} along with properties from Dublin
Core~\cite{weibel1998dublin}. In addition, we used the RDF
version~\cite{mccrae2015ontology} of the META-SHARE
model~\cite{gavrilidou2012meta}, to provide for metadata properties that are
specific to language data and linguistic research. As such, in this paper we
describe the creation of the largest collection of information about language
resources and briefly describe its publication on the Web by means of linked
data principles.

The rest of the paper is structured as follows...

\section{Related Work}

\section{Extraction of data}

\section{Harmonization and duplicate detection}

\section{The Linghub portal}

\section{Conclusion}


\section*{Acknowledgments}

%LingHub was made possible due to significant help from a large number of people,
%in particular we would like to thank the following people: Benjamin Siemoneit
%(Bielefeld University), Tiziano Flati (University of Rome, La Sapienza), Martin
%Br\"ummer (University of Leipzig), Sebastian Hellmann (University of Leipzig),
%Bettina Klimek (University of Leipzig), Penny Labropolou (IEA-ILSP), Juli
%Bakagianni (IEA-ILSP), Stelios Piperidis (IEA-ILSP), Nicoletta Calzolari
%(ILC-CNR), Riccardo del Gratta (ILC-CNR), Marta Villegas (Pompeu Fabra),
%N\'uria Bel (Pompeu Fabra) and Christian Chiarcos (Goethe-University Frankfurt).
%
%+Asun

\bibliographystyle{acl}
\bibliography{../linghub-acl15}

\end{document}
