\documentclass[11pt]{article}
\usepackage{acl2015}
\usepackage[utf8]{inputenc}
\usepackage{times}
\usepackage{url}
\usepackage{latexsym}

\usepackage{makecell}
\usepackage{pbox}
\usepackage{color}
\usepackage{graphicx}

\setlength\titlebox{6cm}

% You can expand the titlebox if you need extra space
% to show all the authors. Please do not make the titlebox
% smaller than 5cm (the original size); we will check this
% in the camera-ready version and ask you to change it back.


\title{Reconciling Heterogeneous Descriptions of Language Resources}

\author{John P. McCrae, Philipp Cimiano \\
  CIT-EC, Bielefeld University \\
  Bielefeld, Germany \\
  {\scriptsize\tt\{jmccrae, cimiano\}@cit-ec.uni-bielefeld.de} \\
  {\bf Victor Rodr\'iguez Doncel, Daniel Vila-Suero}\\
  {\bf Jorge Gracia} \\
  Universidad Polit\'ecnica de Madrid \\
  Madrid, Spain \\
  {\scriptsize\tt\{vrodriguez, dvila, jgracia\}@fi.upm.es} \\\And
  Luca Matteis, Roberto Navigli \\
  University of Rome, La Sapienza \\
  Rome, Italy \\
  {\scriptsize\tt \{matteis, navigli\}@di.uniroma1.it} \\
  {\bf Andrejs Abele, Gabriela Vulcu}\\
  {\bf Paul Buitelaar} \\
  Insight Centre, National University of Ireland\\
  Galway, Ireland \\
  {\scriptsize\tt \{andrejs.abele, gabriela.vulcu,}\\
  {\scriptsize\tt paul.buitelaar\}@insight-centre.org} \\}
 
\date{}

\begin{document}
\maketitle
\begin{abstract}
        Language resources are a cornerstone of linguistic research and for the
        development of natural language processing tools, 
	but the discovery of relevant resources remains a challenging task.
        This is due to the fact that relevant metadata records are spread among
        different repositories and it is currently impossible to query all these
        repositories in an integrated fashion, as they use different data
        models and vocabularies. In this paper we present a first attempt to
        collect and harmonize the metadata of different repositories, thus
        making them queriable and browsable in an integrated way. We make use
        of RDF and linked data technologies for this and provide a first level
        of harmonization of the vocabularies used in the different resources by
        mapping them to standard RDF vocabularies including Dublin Core and
        DCAT. Further, we present an approach that relies on NLP and in
        particular word sense disambiguation techniques to harmonize resources
        by mapping values of attributes -- such as the type, license or intended
        use of a resource -- into normalized values. Finally, as there are
        duplicate entries within the same repository as well as across different
        repositories, we also report results of detection of these duplicates.
\end{abstract}

\section{Introduction}

Language resources are the cornerstone of linguistic research as well as of
computational linguistics. Within NLP, for instance, most tools developed
require a corpus to be trained (e.g. language models, statistical taggers,
statistical parsers, and statistical machine translation systems) or they
require lexico-semantic resources as background knowledge to perform some task
(e.g. word sense disambiguation). 
As the number of language resources available keeps growing, the task of
discovering and finding resources that are pertinent to a particular task
becomes increasingly difficult. While there are a number of repositories that
collect and index metadata of language resources, such as
META-SHARE~\cite{federmann2012meta},
CLARIN~\cite{broeder2010data}, LRE-Map~\cite{calzolari2012lre},
Datahub.io\footnote{\url{http://datahub.io/}} and OLAC~\cite{simons2003open}, they do not provide a complete
solution to the discovery problem for two reasons. First, integrated search over
all these different repositories is not possible, as they use different data
models, different vocabularies and expose different interfaces and APIs. 
Second, these repositories must strike a balance between quality and coverage,
either opting for coverage at the expense of quality of metadata, or \emph{vice
versa}.

When collecting metadata from multiple resources, we understand that there are
two principal challenges: property harmonization and duplication detection. Harmonization
is the challenge of verifying that there is not only structural and syntactic 
interoperability between the resources in that they use the same property, for
example Dublin Core's language property, but also that they use the same value.
For example, the following values of the language property are likely to be
equivalent: ``French'', ``Modern French'', ``fran\c{c}ais'', ``fr'', ``fra'' and
``fre''. It is difficult to write queries on a dataset if every property has
many equivalent values and thus it is essential to use a single representation.
Secondly, we wish to detect duplicates that occur either due to the original
representation or from multiple sources. It is clear that if a large number of
records in fact describe the same resource then queries for that resource will
return too many resources that may lead to errors (or annoyance) for users. For
example, the ``Universal Declaration of Human Rights'' is available in 444
languages\footnote{\url{http://www.ohchr.org/en/udhr/pages/introduction.aspx}}
and listing each translation as a single resource (as the CLARIN VLO does) does  not correctly capture the nature of the resource. Furthermore, these resources may not match some queries, such as
for example `resources in more than one language', and as such it is preferable to
merge these individual records into a single complex record.
%In this paper we address the challenge of harmonizing the metadata originating
%from the above mentioned repositories and propose an approach that attempts to
%maximize coverage by indexing the union of the entries from all these
%repositories, while at the same time ensuring the quality of the data by mapping
%the values of attributes into a normalized representation that supports querying
%and browsing data across the different repositories. Our methodology is as
%follows:
%
%\begin{enumerate}
%\item \textbf{Metadata collection:} We collect the metadata records from four
%    repositories: META-SHARE, CLARIN, LRE-Map and Datahub.io. These resources
%    are very
%different in terms of their size and coverage, and most importantly use very
%different data models.
%\item \textbf{Schema Matching:} To enable these resources to be represented using the same metadata
%model, we apply transformation scripts to the different datasets to transform
%their metadata into RDF using DCAT~\cite{maali2014data}, Dublin
%Core~\cite{weibel1998dublin} and DataID ~\cite{brummer2014dataid} for a basic
%harmonization between resources at the schema level. 
%\item \textbf{Metadata harmonization:} Striving for a stronger harmonization at
%    the data level such that resources are not only described by the same
%    attributes but also use comparable values for those attributes, we apply word sense
%    disambiguation techniques to normalize values of the data to support
%    cross-repository harmonization and thus integrated querying of the data.% In harmonizing the data, we focus on four attributes, i.e. \emph{language}, \emph{resource type}, \emph{rights} and \emph{availability}.
%\item \textbf{Duplicate detection:} We apply duplicate detection techniques to
%    find duplicates both within one and across different repositories.
%%\item \textbf{Publication:} We make the harmonized dataset available through a portal called LingHub that is both accessible to machines as it builds on Linked Data principles and releases a SPARQL endpoint, as well as to humans as it can be browsed via a standard browser.
%\end{enumerate}

%\textcolor{red}{JG: these items should map better with the subsequent sections (which are metadata collection, metadata cleaning, and duplicate detection)}

%\textcolor{red}{JG: shall we add the URL of lingub, or better not because of anonymity?}

As the main contribution of this paper, we present the methods used to harmonize
data across repositories. Due to the different kinds of values and target
taxonomies chosen for each property, these methods vary but all are based on
state-of-the-art NLP techniques, including word sense disambiguation, and make
major improvements to the data quality of our metadata records.
%In particular, we present methods that rely on state-of-the-art word sense disambiguation algorithms to map values of attributes to a normalized representation. We apply this method to normalize the values of attributes such as language and type. We show that this is an effective method for normalizing the data. 
Second, we show indeed that duplicate metadata records are pervasive and that
they occur both within and across repositories. We then present a simple yet
effective approach to detect duplicates within and across repositories. 

The paper is structured as follows: we give an overview of work related to
harmonization of data as well as an overview of existing metadata repositories
for linguistic data in Section \ref{rel_work}. We describe our metadata
collection and schema matching strategy in Section \ref{collection}. We describe
our techniques for metadata harmonization in Section \ref{harmonization}. We
describe our methods for duplication detection in Section \ref{duplicates}. The
performance of the different techniques is reported in each of these sections.
We discuss our methodology and approach from a wider point of view in Section
\ref{discussion}. 


\section{Related Work}

\label{rel_work}

Interoperability of metadata is an important problem in many domains and
harmonizing schemas from different sources has been recognized as a major
challenge 
\cite{nilsson2010interoperability,khoo2010merging,nogueras2004metadata}. 
There are different approach to data integration. One approach consists on
mapping data to one \emph{monolithic} ontology that needs to be general enough
to accommodate all the data categories from different sources. While this is 
appealing as it supports integrated querying of data, a
single ontology cannot predict all aspects of metadata
records, that all users may wish to record. In contrast, the linked data approach relies on
multiple, standardized smaller and reusable vocabularies, each representing a 
subset of the data.  In this line, some experts have recommended \cite{brooks2006towards}:

\begin{quote}
``A larger set of ontologies sufficient for particular purposes should
be used instead of a single highly constrained taxonomy of values.''
\end{quote}

%Many of these communities have independently adopted RDF models to enable open
%schemas, and there have even been attempts to generate metadata in RDF
%automatically~\cite{jenkins1999automatic}. 


In the context of linguistic data, different approaches have been pursued to collect metadata of resources. 
Large consortium-led projects and initiatives such as the CLARIN projects and
META-NET have attempted to create metadata standards for representing linguistic data. 
Interoperability of the data stemming from these two repositories is however
severely limited due to incompatibilities in their data models.
META-SHARE favors a qualitative  approach in which a relatively complex XML
schema is provided to describe metadata of resources ~\cite{gavrilidou2012meta}.
At the same time, considerable effort has been devoted to ensuring data quality
~\cite{piperidis2012meta}.
In contrast, CLARIN does not provide a single schema, but a set of `profiles'
that are described in a schema language called the \emph{CMDI Component
Specification Language}~\cite{broeder2012cmdi}. Each institute describing
resources using CMDI can instantiate the vocabulary to suit their particular
needs. Similarly, an attempt has been made to catalogue language resources by
assigning them a single unique identifier~\cite{choukri2012using}.

Other more decentralized approaches are found in initiatives such as the
LRE-Map~\cite{calzolari2012lre} which provides a repository for researchers
who want to submit the resources accompanying papers submitted to conferences.
Most fields in LRE-Map consist of a text field with some prespecified options to
select and a thorough analysis of the results has been
conducted~\cite{mariani2014facing}.

Similarly, the \emph{Open Linguistics Working Group}~\cite{chiarcos2012open} has
been collecting language resources published as linked data in a crowd-sourced
repository at Datahub.io, in order to monitor the \emph{Linguistic Linked Data
cloud} and produce a diagram showing the status of these resources.

This clearly shows that the field is very fragmented, with different
players using different approaches and most importantly different meta- and 
data models, thus impeding the discovery and integration of linguistic data. 

\section{Metadata collection and Schema Matching}

\label{collection}

\begin{table}
\resizebox{0.80\textwidth}{!}{\begin{minipage}{\textwidth}
    \begin{tabular}{l|r|r|r}
        \thead{Source} & \thead{Records} & \thead{RDF Triples} & \thead{Triples per \\ Record} \\
        \hline                                              
        META-SHARE          &   2,442 &   464,572 & 190.2 \\
        CLARIN VLO          & 144,570 & 3,381,736 &  23.4 \\
        Datahub.io          &     218 &    10,739 &  49.3 \\
        LRE-Map (LREC 2014) &     682 &    10,650 &  15.6 \\
        \hline
        LRE-Map (Non-open)  &   5,030 &    68,926 &  13.7 \\
% OLAC size is estimated must be calculated correctly before final version
        OLAC                & 217,765 & 2,613,183 &  12.0 \\
        ELRA Catalogue      &   1,066 &    22,580 &  21.2 \\
        LDC Catalogue       &     714 &       n/a &   n/a \\
        \hline
    \end{tabular}
\end{minipage}}
    \caption{\label{tab:resource-sizes}The sizes of the resources in terms of
    number of metadata records and total data size}
\end{table}

In this section we describe the different methods applied to collect metadata from the different repositories:

\begin{itemize}
\item \textbf{META-SHARE:} For META-SHARE, a dump of the data was
provided by the ILSP managing node of the META-NET project in XML format. We developed a custom script to convert this
into the RDF data model, explicitly aligning data elements to the
Dublin Core metadata vocabulary and add these as extra RDF triples to the root
of the record. Frequently, these properties were deeply nested in the XML file
and manual analysis was required to detect which instances truly applied to the
entire metadata record.

\item \textbf{CLARIN:} For CLARIN, we rely on the OAI-PMH~\cite{sompel2004resource} framework to harvest data.
The harvested OAI-PMH records comprise a header with basic information as well as a download link and a secondary XML description
section that is structured according to the particular needs of the data provider.
So far, we limit ourselves to collecting only those records that
have Dublin Core properties.
\item \textbf{LRE-Map:} For LRE-Map we used the available RDF/XML data
    dump\footnote{\url{http://datahub.io/organization/institute-for-computational-linguistics-ilc-cnr}},
    which contains submission information from the LREC 2014 conference, as well as data
    from other conferences, which is not freely available.
    In the RDF data, we gathered additional
    information about language resources, including the title of the paper 
    describing the resource.
\item \textbf{Datahub.io:} The data from Datahub.io was collected by means of
    the CKAN API\footnote{\url{http://datahub.io/api/3/} documented at
    \url{http://docs.ckan.org/en/latest/api/index.html}}. As Datahub.io
is a general-purpose catalogue we limited ourselves to extracting only those
resources that were of relevance to linguistics. For this, we used an existing
list of relevant categories and tags maintained by the Working Group on Open
Linguistics~\cite{chiarcos2012open}. The data model used by Datahub.io is 
also based on DCAT, so little adaptation of the data was required.
\item \textbf{OLAC:} The Open Language Archives Community also relies on OAI-PMH
    to collect metadata and overlaps significantly with the CLARIN VLO.
    Unfortunately the data on this site is not openly licensed.
\item \textbf{ELRA and LDC Catalogues:} These two organizations sell language
    resources and their catalogues are available online. The metadata records are
    not themselves openly licensed.
\end{itemize}

The total size in terms of records and triples (facts) as well as the average
number of triples per repository are given in Table \ref{tab:resource-sizes}, where
we can see significant differences in size and complexity of the resources. Note
for the rest of this paper we will concern ourselves only with the openly
licensed resources.

\section{Metadata harmonization}

\label{harmonization}

As metadata has been obtained from different repositories, there are many
incompatibilities between the values used in different resources. 
While some repositories ensure high-quality metadata in general, we also
discovered inconsistencies in the use of values. For instance, while META-SHARE
recommends the use of ISO 639-3\footnote{\url{http://www-01.sil.org/iso639-3/}} tags for
languages, a few data entries use English names for the language
instead of the ISO code. We describe our approach to data value normalization
below. In this initial harmonization phase we focused on the key
questions of whether a resource is available, that is the given URL
resolves, and whether the terms and conditions under which the resource can be
used are specified. Further, we consider three key aspects that users need to
know about resources to help them decide whether the resource matches their
needs, namely: the type of the resource (corpus, lexical resource, etc.),
 intended use of the resource and languages covered. We note that many resources
have multiple values for the same property (e.g., language), thus we allow
multiple values at the record level, while still permitting more specific
annotation deeper in the record.

\subsection{Availability}

In order to enable applications to (re)use language resources, we should find
out if the resources described can still be accessed. For this we focused on the
properties which were mapped to DCAT's `access URL' property in the previous
section. These `access URLs' are intended to refer to HTML pages
containing either download links or information on how to retrieve and use the
resource. We augment the data with information about which links are valid and
about the form of the content returned (e.g. HTML, XML, PDF, RDF/XML, etc.).
Therefore, as we deal with heterogeneous sources and repositories, we 
analyzed access related characteristics and initially focused on answering two 
questions: \textit{Is the language resource available and accessible on the Web
and in what format?}.

To assess the current situation, we crawled and performed an analysis on a set of 
119,290 URLs~\footnote{Due to crawling restrictions, only 60\% of the URLs of the dataset were actually crawled}.
%corresponding to more than 60\% of the complete set of access 
%URLs\footnote{We collected and harmonized 191,006 URLs but we filtered out the URLs of hd.handle.net from the sample due to crawling restrictions for this domain}\footnote{The complete set of raw results can be accessed here: \url{http://bit.ly/1GxqJI7}}. 
Our analysis showed that more than 95\% of the URLs
studied corresponded to accessible URLs (i.e., HTTP Response Code 200 OK), which
indicates that in a high number of cases at least some information is provided
to potential consumers of the resource.

Furthermore, our
assessment showed that more than 66\% of the accessible URLs corresponds to HTML
pages, around 10\% to RDF/XML documents, and other non-text
formats sum up to almost 10\% of the URLs analyzed 
(see Table \ref{tab:formats}). It is important to note that these results
only describe what was returned by the service, and do not well reflect the
actual format or availability of the data. For example, the high number of
resources returning RDF/XML is mostly due to two CLARIN contributing institutes
adopting RDF for their metadata.

%\begin{itemize}
%	\item \textbf{META-SHARE:} In META-SHARE, more than 95\% of the URLs corresponded to HTML pages and almost 5\% to PDF documents.
%	
%	\item \textbf{CLARIN:} In CLARIN, 65\% of the URLs corresponded to HTML pages but we found higher format heterogeneity (JPEG, XML, PDF, compression formats, etc.) and more than 10\% of RDF/XML documents.
%	
%	\item \textbf{Datahub.io:} Datahub.io is not only the most heterogeneous source with 19 different formats but also the one providing more machine-processable formats with only 45\% of HTML pages and with almost 30\% of documents in different RDF serializations (mainly RDF/XML and Turtle).
%	
%	\item \textbf{LRE-Map:} Finally, LRE-Map is widely dominated by HTML pages (more than 97\%).
%	
%\end{itemize}

\begin{table}
    \begin{center}
	\begin{tabular}{l|cc}
            Format   & Resources  & Percentage\\
		
		\hline                                              
                HTML                &	67,419 & 66.2\%\\
                RDF/XML             &	9,940  & 9.8\% \\
                JPEG Image          &   6,599  & 6.5\% \\
                XML (application)   &	5,626  & 5.6\% \\
                Plain Text          & 4,251    & 4.2\% \\
                PDF                 &	3,641  & 3.6\% \\
                XML (text)          & 3,212    & 3.2\% \\
                Zip Archive         &	801    & 0.8\% \\
                PNG Image           & 207      & 0.2\% \\
                gzip Archive        & 181      & 0.2\% \\
	\end{tabular}
    \end{center}
	\caption{\label{tab:formats}The distribution of the 10 most used formats within the
        analyzed sample of URLs. Note XML is associated with two MIME types.}
\end{table}

\subsection{Rights}

Language resources are generally protected by copyright laws and they cannot be
used against the terms expressed by the rights holders. These terms of use
declare the actions that are authorized (e.g. derive, distribute) and the
applicable conditions (e.g. attribution, the payment of a fee). They are an
essential requirement for the reuse of a resource, but their automatic retrieval and processing is
difficult because of the many forms they may adopt: rights information can
appear either as a textual notice or as structured metadata, can consist of a
mere reference to a well-known license (like an Open Data Commons or Creative
Commons license), or it can point to an institution-specific document in a
non-English language. These heterogeneous practices prevent the automated
processing of licensing information. 

%In the four studied respositories licensing
%information is described as follows: 

%\textcolor{red}{THIS PART CAN BE REMOVED WITHOUT SIGNIFICANT LOSS OF INFORMATION. REMOVE THIS IF YOU FIND THE DESCRIPTION TOO LONG}
%\begin{description}
%	\item[META-SHARE] Besides the mere reference to a Creative Commons license, META-SHARE recommends using one of their pre-defined legal texts\footnote{\url{http://www.meta-net.eu/meta-share/licenses} }. These texts are applicable in a number of situations and they foresee variate restrictions, such as limiting the user's nature (academic / non-academic), the user's intent (commercial or non-commercial) or its META-SHARE membership among others. The META-SHARE schemata provide also elements for the fine-grain description of these license terms.
%	\item[CLARIN] Clarin defines three kinds of licenses: those which allow the resource to be publicly distributed, those accesible only for research purposes and finally those which impose additional restrictions or whose use require additional constent from the rights holder. Specific rights and conditions are also available.
%	\item[Datahub.io] Datasets in Datahub.io are qualified in different manners. A license can be chosen from a list when registering a new dataset, but it can also be found in the VoID description of the dataset or even within the dataset itself. Only explictly given rights information was considered when harvesting the resources.
%  \item[LRE-Map] Resources are classified according to their availability (e.g. freely available or from the proprietary's web), and the license description is limited to a simple string text.
%\end{description}
%\textcolor{red}{END OF OPTIONAL PART}
Several challenges are posed for the harmonisation of the rights information: first,
information is often not legally specified but instead vague
statements such as `freely available' are used; second, description
of specific rights and conditions of each license requires complex modelling; 
and finally, due to the sensitivity of the information, only high precision
approaches should be applied.
%Solving these challenges would enable the response to search-by-license queries
%(e.g. '\textit{Which german dictionaries can I use in commercial
%applications?}') across different repositories and it would also ease the task
%of determining the possible compatible licenses for an aggregation of
%heterogenously licensed resources. 

From the RDF License dataset \cite{rdflicense} we extracted the  title, URI and
abbreviation of the most
commonly used licenses in different forms, and searched for exact matches
normalizing for case, punctuation and whitespace. This introduced some errors due to
dual-licensing schemes or misleading description were introduced.
We manually evaluated all matching licenses and found 95.8\% of the
recognised strings were correctly matched. %3 between 71 
With this approach we could identify matching licenses for only 1\% of the
metadata entries. However, 
our observations suggest that this is due to the uninformative content for the
license attribute. Furthermore, we note that more sophisticated
methods have been shown to improve recall, but they do this at the cost of
precision~\cite{cabrio2014these}.

%For this 
%However, the overall recall was very low because a large number of terms and
%conditions were specific and sometimes in other languages than English. Other
%NLP techniques might be applied for extracting this information but this
%sensitive information requires a good precision rate that existing algorithms do
%not provide. %I CAN CITE THIS TWO WORKS TO SUPPORT THIS IF NEEDED ada
%Finally, some of the most common licenses present in the META-SHARE and CLARIN
%have been mapped and updated into the RDF Dataset. dafdaf
%%I GUESS WE CANNOT GIVE HERE URLS IF WE WANT TO HIDE OUR IDENTITY...

\subsection{Usage}

The language resource usage indicates the purpose and application for which 
the LR was created or which it has since be used. 
For META-SHARE we rely on the 83 values of the 
{\tt useNLPSpecific} property and for LRE-Map we have a more limited list 
of 28 suggested values and many more user-provided free text entries, 3,985 in
total (no other source contained this information). We manually mapped the 28
predefined values in LRE-Map to one of the 83 values predefined in META-SHARE.
For the user-provided intended usage values, we developed a matching algorithm
that identifies the corresponding META-SHARE intended use values. First we
tokenized the expressions, then we stemmed the tokens using the Snowball
stemmer~\cite{porter2001snowball}, and we performed a string inclusion match, i.e. checking whether
META-SHARE usages are included in the free text entries. For some entries we
retrieved several matches (e.g.  `Document Classification, Text categorisation'
matched both `document classification' and `text categorisation'), assuming that
in the case of multiple matches the union of the intended usages was meant.
With this algorithm we identified 66 matches on a random sample of 100
user-provided entries and they were all correct matches. From the remaining 34
unmatched entries, 16 were empty fields or non specific e.g.  `not applicable',
`various uses'. Other 16 entries were too general to be mapped to an intended
use defined in the META-SHARE vocabulary e.g. `testing', `acquisition'. We had
one false negative `taggin pos'[sic] and one usage that is not yet in META-SHARE
`semantic system evaluation'. On this basis we had 98-99\% accuracy on the
results.  Following the aforementioned algorithm we identified 65\% matches on
the entire set of user-entries. We further investigated the remaining 35\%
non-matches and we identified further intended use values that are not yet in
META-SHARE vocabulary, e.g.  `entity linking', `corpus creation', which we will
suggest as extensions of the META-SHARE vocabulary.

    
\subsection{Language}

To clean the names of languages contained in metadata records, we aligned
to the ISO 639-3 standard. First we extracted all the language labels
from our records and obtained a total of 833 distinct language labels. Next we
leveraged two resources to map these noisy language labels to standard ISO
codes: (i) the official SIL
database\footnote{\url{http://www-01.sil.org/iso639-3/download.asp}}, which
contains all the standard ISO codes and their \textit{English} names, and (ii)
BabelNet\footnote{\url{http://babelnet.org/}} \cite{NavigliPonzetto:12aij}, a
large multilingual lexico-semantic resource containing, among others,
translations and synonyms of various language names along with their ISO codes. 

To perform the mapping in an automatic manner, we compared each of the 833 noisy
language labels against the language labels contained in SIL and BabelNet using
two string similarity algorithms: the \textit{Dice coefficient} string
similarity algorithm and the \textit{Levenshtein} distance string metric. 


\begin{table}
\resizebox{0.92\textwidth}{!}{\begin{minipage}{\textwidth}
    \begin{tabular}{l|c|cc|cc}

Input & \thead{Expected \\ output}  & \multicolumn{2}{c|}{\thead{BabelNet \\ output}} & \multicolumn{2}{c}{\thead{SIL \\ output}} \\
& & \tiny\textit{dice} & \tiny\textit{leven} & \tiny\textit{dice} & \tiny\textit{leven}\\
        \hline         
Kurdish & kur & kur & kur & kur & kur \\    
\tiny\textit{rank -- distance} & & \tiny 1 & \tiny 0 & \tiny 1 &\tiny 0\\
\tiny\textit{label} & & \tiny Kurdish & \tiny Kurdish & \tiny kurdish & \tiny Kurdish\\
\hline
Bokmål  & nob & nob & nob & bok* & bdt*\\
\tiny\textit{rank -- distance} & & \tiny 1 & \tiny 0 & \tiny 0.57 &\tiny 3\\
\tiny\textit{label} & & \tiny bokmål & \tiny Bokmål & \tiny bok & \tiny Bokoto\\                               
\hline
\pbox{20cm}{Ñahñú \\ (Otomí)}  & oto & omq* & otm* & ttf* & las*\\
\tiny\textit{rank -- distance} & & \tiny 0.38 & \tiny 8 & \tiny 0.35 &\tiny 7\\
\tiny\textit{label} & & \tiny \pbox{20cm}{Otomí\\ Mangue} & \tiny \pbox{20cm}{Eastern\\ Otomí} & \tiny tuotomb & \tiny \pbox{20cm}{lama\\ (togo)}\\     
\hline
\pbox{20cm}{Turkish \\(Türkçe)}  & tur & tur & tur & tur & crh*\\
\tiny\textit{rank -- distance} & & \tiny 0.7 & \tiny 6 & \tiny 0.7 &\tiny 7\\
\tiny\textit{label} & & \tiny Turkish & \tiny \pbox{20cm}{Türkiye\\Türkçesi} & \tiny turkish & \tiny \pbox{20cm}{Turkish,\\ Crimean}\\                            
    \end{tabular}
    
\end{minipage} }
    \caption{\label{tab:lang-io}Excerpt output of language mapping. \newline * indicates mismatches.}
\end{table}

%The Dice coefficient algorithm takes two strings as inputs and returns a value between 0 and 1 which we call \textit{rank}, where values closer to 0 represent lower similarity and values closer to 1 represent higher similarity. The Levenshtein metric instead returns a value which we call \textit{distance}, and is the number of deletions, insertions, or substitutions required to transform the first input string into the second. Therefore for Levenshtein, a value closer to 0 represents higher similarity. 

Table \ref{tab:lang-io} reports an excerpt of the results showcasing in the
first row a
match for all cases, in the second  a match for BabelNet but not for SIL, and
in the third
a mismatch for all. Furthermore, the final row reports a mismatch from Levenshtein,
where `Turkish, Crimean' is matched instead.

In order to measure the accuracy of each approach we tested the mapping
algorithms against a manually annotated dataset containing 100 language labels
and ISO codes.
In Table \ref{tab:language-comparison}, we present the accuracy of our methods
based on the number of labels correctly identified (``label accuracy'') and
the accuracy weighted for the number of
metadata records with that label (``instance accuracy'').
The best results are
obtained using BabelNet as the source of language labels. BabelNet is more
accurate in matching language labels, largely because it contains translations,
synonyms and obsolete spellings of most, even rare or dialectal, languages. SIL
on the other hand only contains the English representation of each ISO code,
failing to induce certain mappings. Furthermore, the Dice coefficient string
similarity algorithm yields more accurate results compared to the Levenshtein
distance metric. We hypothesize that this is mainly due to the fact that the
Dice coefficient is more lenient compared to the Levensthein metric as it is
insensitive to the order of words. For instance, using Dice coefficient, the
input label `Quechua de Cotahuasi (Arequipa)' matches `Cotahuasi Quechua'
correctly. With the Levenshtein algorithm, however, using the same input as
earlier, the label `Quechua cajamarquino' is mistakenly matched instead.


\begin{table}
\resizebox{1.1\textwidth}{!}{\begin{minipage}{\textwidth}
    \begin{tabular}{l|c|c}
        Resource   & \thead{Label\\Accuracy}  & \thead{Instance\\Accuracy} \\
%    \renewcommand{\arraystretch}{0.5}\begin{tabular}[x]{@{}c@{}}\tiny
%    Instance\\Accuracy\end{tabular} \\%& Average\\
        \hline                                              
        SIL \small\textit{dice coefficient}  & 81\% & 99.50\% \\%& 0.93 \small\textit{rank}  \\
        SIL \small\textit{levenshtein}  & 72\% & 99.42\% \\%& 0.80 \small\textit{distance} \\
        BabelNet \small\textit{dice coefficient} & \textbf{91\%} & 99.87\% \\%& 0.97 \small\textit{rank} \\
        BabelNet \small\textit{levenshtein} & \textbf{89\%} & 99.85\% \\%& 1.20 \small\textit{distance} \\
        \hline
        SIL + BabelNet & & \\
        \small\textit{dice coefficient} & \textbf{91\%} & 99.87\% \\%& 0.97 \small\textit{rank} \\
        \small\textit{levenshtein} & \textbf{89\%} & 99.85\% \\%& 1.01 \small\textit{distance} \\
    \end{tabular}
    \end{minipage} }
    \caption{\label{tab:language-comparison}Accuracy of language mappings}
\end{table}

Overall, combining BabelNet and SIL yields the same normalization accuracy as
BabelNet alone. Nonetheless, we can observe a slight decrease in the average
distance returned by the Levensthein algorithm. The addition of a multilingual
semantic database, such as BabelNet, positively affects the ability to match
obsolete names in different languages. 

\subsection{Type}

The type property is used primarily to describe the kind of resource being
described. For META-SHARE, we can rely on the structure of resources to extract
one of four primary resource types, namely, `Corpus', `Lexical Conceptual
Resource', `Lexical Description' and `Tool Service'. However, for the other
sources considered in this paper the type field permits free text input. In order 
to enable users to query resources by type we ran the Babelfy entity linking
algorithm~\cite{Moroetal:14tacl} to identify entities in the string and
then manually selected elements from this list of entities that described the kind of
resource, such as `corpus'. In this way we extracted, 143 categories for
language resources while still ensuring that syntactic variations were accounted
for. The top 10 categories extracted in this way were: `Sound', `Corpus',
`Lexicon', `Tool' (software), `Instrumental Music'\footnote{These
resources are in fact recordings of singing in under-resourced languages}, `Service', `Ontology', `Evaluation',
`Terminology' and `Translation software'.

\section{Duplicate detection}

\label{duplicates}

\begin{table}
    \begin{center}
    \begin{tabular}{l|cc}
        Resource   & \thead{Duplicate \\ Titles} & \thead{Duplicate \\ URLs} \\
        \hline                                                            
        CLARIN{\tiny~(same contributing institute)}     & 50,589           & 20          \\   
        Datahub.io & 0                & 55             \\
        META-SHARE & 63               & 967            \\
    \end{tabular}
    \end{center}
    \caption{\label{tab:self-dupes}The number of intra-repository duplicate labels and URLs for
    resources}
\end{table}

As we are collecting and indexing metadata records from different repositories,
it is possible to find duplicates, that is records that describe the same actual
resource.
In fact, duplicate entries did not only occur across repositories
(we dub these \emph{inter-repository duplicates}) but also within the same
resource (referred to as \emph{intra-repository duplicates}). 
We expand the
definition of inter-repository by noting that CLARIN 
is sourced from a number of different contributing institutes and there are
duplicates between institutes, thus we consider links between records
of different CLARIN institutes as \emph{inter-repository}. Similarly, there has
been no attempt to manage duplicates in LRE-Map and so we handle all links
between LRE-Map records as \emph{inter-repository}.

In order to detect duplicates, we rely on two properties
that should be unique across entries, that is the title and the `access URL'.
%\footnote{Our data model distinguishes
%    between the `access URL', which is normally the (HTML) download page and the
%    `download URL' which is a direct link to the resource. However, this
%    distinction is not generally made by all resources and as such all
%download URLs are treated as access URLs}. We were surprised 
%to discover a significant number of intra-repository duplicates with this heuristic.
In Table
\ref{tab:self-dupes} we show the number of records with duplicate titles or URLs.
Manual inspection of these duplicates yielded the following
observations:

\begin{description}

	\item[META-SHARE] META-SHARE contains a number of duplicate titles.
            However, these title duplicates seem to be errors in the export and can thus be easily corrected. 

    \item[CLARIN] Many resources in CLARIN are described across many records.
        For example, in CLARIN there may be one different metadata record for each chapter of a book or
        recording within an audio or television collection, or in at least one case
        (``The Universal Declaration of Human Rights'') a record exists for each
        language the resource is available in. Thus, we decided to merge the entries which
        share the same title and same contributing institute in CLARIN.
        
    \item[Datahub.io] The creation method of DataHub prevents the creation of
        different entries with the same title, so duplicate titles do not occur
        in the data. However, we found a number of entries having the same
        download URL. This is due to the fact that different resources share
        SPARQL endpoints or download pages, but the records did not describe the
        same resource and so we did not merge these resources.
    
\end{description}

\begin{table*}
    \begin{center}
        \begin{tabular}{ll|ccc}
        Resource    & Resource    & Duplicate Titles & Duplicate URLs & Both \\
        \hline                                                                  
        CLARIN      & CLARIN{\tiny~(other contributing institute)}      & 1,202            & 2,884          & 0    \\
        CLARIN      & Datahub.io  & 1                & 0              & 0    \\
        CLARIN      & LRE-Map     & 72               & 64             & 0    \\
        CLARIN      & META-SHARE  & 1,204            & 1,228          & 28    \\
        Datahub.io  & LRE-Map     & 59               & 5              & 0    \\
        Datahub.io  & META-SHARE  & 3                & 0              & 0    \\
        LRE-Map     & LRE-Map     & 763              & 454            & 359  \\
        LRE-Map     & META-SHARE  & 91               & 51             & 0    \\
        \hline
        All         & All         & 3,395            & 4,686          & 387  \\
        \end{tabular}
    \end{center}
    \caption{\label{tab:dupes}Number of duplicate inter-repository records by type}
\end{table*}

Table \ref{tab:dupes} shows the number of resources with the same title
(Duplicate Titles), same URL (Duplicate URLs) as well as same title \textbf{and}
same URL within and across repositories.  We apply the following strategy in
handling duplicates:

\begin{description}
    \item[Intra-repository duplicates] As intra-repository duplicates are
        mostly either system errors or series of closely related resources, we
        simply merge the corresponding metadata entries. If a property is one-to-one we take only the
        first value.
    \item[Inter-repository duplicates] Inter-repository duplicates represent
        multiple records of the same underlying resource, they are linked to one
        another by the `close match' property.
\end{description}

\emph{Note we do not remove duplicates from the dataset we either combine them
into a more structured record or mark them as deprecated.}

%\begin{enumerate}
%\item Remove intra-repository duplicates from METASHARE as well as metadata
%records corresponding to series of data.
%\item Remove inter-repository duplicates.
%\item Remove intra-repository duplicates from LRE-Map and CLARIN as they contain a higher number of duplicates due to the fact that the data originates from different data contributors.
%\end{enumerate}

%\textcolor{red}{JG: Is the order important? Why inter-repositories are in the middle?}

We evaluate the precision of this approach on a sample of 100 inter-repository entries identified
as duplicates according to the above mentioned approach. We manually classify
the matches into \emph{correct}, \emph{incorrect} as well as \emph{unclear}, if
there was insufficient information to make a decision, the resources overlapped
or were different versions of each other. Table \ref{tab:dupe-precision} shows
these results. We see that with 99\% precision the method identifying duplicates
if both title and URL match yields the best results. While the recall is
difficult to assess, an analysis of the data quickly reveals that there are many
duplicates not detected using this method. For example, for the Stanford Parser~\cite{de2006generating},
we find metadata records with all of the following titles: ``Stanford Parser'',
``Stanford Dependency Parser'',  ``Stanford Lexicalized Parser'', ``Stanford's
NLP Parser'', ``The Stanford Parser'', ``The Stanford Parser: A Lexicalized
Parser''.



\begin{table}
    \begin{tabular}{l|ccc}
        Duplication & Correct & Unclear & Incorrect \\
        \hline                    
        Titles      &    86   &   6     &    8      \\ 
        URLs        &    95   &   2     &    3      \\
        Both        &    99   &   1     &    0      \\
    \end{tabular}
    \caption{\label{tab:dupe-precision}Precision of matching strategies from a
    sample of 100}
\end{table}

\section{Discussion}
\label{discussion}

\begin{table}
\resizebox{\columnwidth}{!}{%
    \begin{tabular}{l|cc}
        Property   &  \thead{Record Count\\(As percentage of all records)} & Triples \\
        \hline
        Access URL &  91,615 (91.6\%) & 191,006  \\
        Language   &  50,781 (50.7\%) & 98,267   \\
        Type       &  15,241 (15.2\%) & 17,894   \\
        Rights     &   3,080 (3.0\%)  & 8915     \\
        Usage      &   3,397 (3.4\%)  & 4,530    \\ 
\end{tabular}}
    \caption{\label{tab:total}Number of records and facts harmonized by our
    methods}
\end{table}
        



The rapid developments of natural language processing technologies in the last
few years has resulted in a very large number of language resources being
created and made available on the web. In order to enable these resources to be
reused appropriately it is necessary to properly document resources and make
this available as structured, queriable metadata on the Web. Current approaches
to metadata collection are either \emph{curatorial}, where dedicated workers
maintain metadata of high quality, such as the approach employed by META-SHARE.
This approach ensures metadata quality but is very expensive and as such it is
unlikely that it will be able to handle the vast number of resources published
every year. In contrast, \emph{crowd-sourced} resources rely primarily on
self-reporting of metadata, and this approach has a high recall but is very
error-prone and this unreliability can be plainly seen in resources such as
LRE-Map. In this paper, we have aimed to break this dichotomy by aggregating
resources from both curated and crowd-sourced resources, and applied natural language
processing techniques to provide a basic level of compliance among these
metadata records, and have achieved this for a large number of records as
summarized in table \ref{tab:total}. In this sense we have considered a small
set of properties that we regard as essential for the description and discovery
of relevant language resource, that is: resource type, language, intended use,
and licensing conditions.  For the
language property we have shown that it can be harmonized across repositories
with high accuracy by mapping values to a controlled vocabulary list, although
the data indicated that there were still many languages which were not covered
in the ISO lists. For the type, rights and usage properties, whose content is
not as limited, it is harder to harmonize but we were still able to show that in
many cases these results can be connected to known lists of values. This is
important as it would allow for easier queries of the resource. 

Besides harmonizing values of data, we see two further key aspects to ensure
quality of the metadata.  First, broken links should be avoided as they are
indicators of low curation and low quality. Thus, we automatically detect such
broken URLs and remove them from the dataset. A second crucial issue is the
removal of duplicates, which are also a sign of low quality.

We have investigated different strategies for detecting duplicates. We observed
that the case in which two metadata records have been provided to different
repositories is common. When integrating data from different repositories, these
entries become duplicated. In other cases, particularly in CLARIN, different
metadata records are created for parts of a resource. Genuine duplication likely
affects about 7\% of records, underlining the value of collecting resources from
multiple sources. We further note that it is important to take a high precision
approach to deduplication as the merging of non-duplicate resources can hide
resources entirely from the query. Thus, we have proposed high-precision methods
for detecting such duplicates. 

Finally, we note that the data resulting from this process is available under
the Creative Commons Attribution Non-Commercial ShareALike License and the data
can be queried through a portal, which is available at \textbf{URL anonymized}.
Furthermore, all code described in this paper is accessible from a popular open
source repository.\footnote{To remain anonymous we cannot include URLs for these
resources at this point}

\section{Conclusion}

We have studied the task of harmonizing records of language resources that are
heterogeneous on several levels and have shown that the application of NLP
techniques allows to provide common metadata that will better enable users to
find language resources for their specific applications. We note that this work
is still on-going and should be improved in not only the accuracy and coverage
of harmonization, but also in the number of properties that are harmonized
(authorship and subject topic are planned). We hope that this new approach to
handling language resource metadata will better enable users to find language
resources and assist in the creation of new domains of study in computational
linguistics.

\section*{Acknowledgments}

LingHub was made possible due to significant help from a large number of
people, in particular we would like to thank the following people: Benjamin
Siemoneit (Bielefeld University), Tiziano Flati (University of Rome, La
Sapienza), Martin Br\"ummer (University of Leipzig), Sebastian Hellmann
(University of Leipzig), Bettina Klimek (University of Leipzig), Penny
Labropolou (IEA-ILSP), Juli Bakagianni (IEA-ILSP), Stelios Piperidis
(IEA-ILSP), Nicoletta Calzolari (ILC-CNR), Riccardo del Gratta (ILC-CNR), Marta
Villegas (Pompeu Fabra), N\'uria Bel (Pompeu Fabra), Asunci\'on G\'omez-P\'erez 
(Universidad Polit\'ecnica de Madrid) and Christian Chiarcos
(Goethe-University Frankfurt).
%
%+Asun

\bibliographystyle{acl}
\bibliography{linghub-acl15}

\end{document}
